\documentclass{article}
\usepackage{mathpartir}
\usepackage{amsmath}
\usepackage{amssymb}
\usepackage{stmaryrd}

\title{HSE+}
\author{Alec Story}

\begin{document}

\newcommand{\union}{\cup}
\newcommand{\Union}{\bigcup}
\newcommand{\intersection}{\cap}
\newcommand{\thickbar}{\talloblong}
\newcommand{\Skip}{\hbox{\bf skip}}

\maketitle

The CHP that existing users expect has too many primitives to model cleanly,
having both shared variables and channels, and the channels support
multi-channel synchronization, and peeking at channel values and state.  Rather
than model CHP directly, we propose modeling a weaker language, HSE+ which
contains only the shared variables and selection statements from CHP, making it
much closer to HSE, although it assumes distinguishable states which HSE does
not.

\section{Grammar}
\begin{align*}
\mathrm{\ell} & :: = \top \; | \; \bot \\
\mathrm{b} & ::= \ell \; | \;  a \; | \;
                 \mathrm{b}_1 \wedge \mathrm{b}_2 \; | \;
                 \mathrm{b}_1 \vee \mathrm{b}_2 \; | \;
                 \lnot \mathrm{b} \\
\mathrm{P} & ::= a := b \; | \; S \; | \; *S \; | \;
                 P_1; P_2 \; | \: P_1 || P_2 \; | \;
                 \mathtt{\Skip} \\
\mathrm{S} & ::=
    [ b_1 \rightarrow P_1  \talloblong \; ... \; \talloblong b_n \rightarrow P_n ] \; | \;
    [ b_1 \rightarrow P_1 | \; ... \; | b_n \rightarrow P_n ]
\end{align*}
\section{Operational Semantics}
\subsection{Booleans}

\begin{mathpar}
\inferrule* [left=Primitive]
    { }
    {\sigma \models \ell = \ell}

\inferrule* [left=Read]
    {a \gets \ell \in \sigma}
    {\sigma \models a = \ell}

\inferrule* [left=Neg]
    {\sigma \models b = \ell}
    {\sigma \models \lnot b = \lnot \ell}

\inferrule* [left=And]
    {\sigma \models b_1 = \ell_1 \\ \sigma \models b_2 = \ell_2}
    {\sigma \models b_1 \land b_2 = \ell_1 \land \ell_2}

\inferrule* [left=Or]
    {\sigma \models b_1 = \ell_1 \\ \sigma \models b_2 = \ell_2}
    {\sigma \models b_1 \lor b_2 = \ell_1 \lor \ell_2}
\end{mathpar}

Behavior of a boolean statement containing an uninitialized variable is
undefined, but if a process is of type $(\emptyset, w)$ (see Type System,
below), then this will never happen.

\subsection{Processes}

\begin{mathpar}

\inferrule* [left=SkipSeq]
    { }
    {\Skip; P, \sigma \rightarrow P, \sigma}

\inferrule* [left=SkipPar]
    { }
    {\Skip || P, \sigma \rightarrow P, \sigma}

\inferrule* [left=ParCommute]
    { }
    {P_1 || P_2, \sigma \rightarrow P_2 || P_1, \sigma}
\end{mathpar}

\begin{mathpar}
\inferrule* [left=StepSeq]
    {P_1, \sigma \rightarrow P_1', \sigma'}
    {P_1; P_2, \sigma \rightarrow P_1'; P_2, \sigma'}

\inferrule* [left=StepPar]
    {P_1, \sigma \rightarrow P_1', \sigma'}
    {P_1 || P_2, \sigma \rightarrow P_1' || P_2, \sigma'}

\inferrule* [left=Assign]
    {\sigma \models b = \ell}
    {a := b, \sigma \rightarrow \Skip, \sigma[ a = \ell]}
\end{mathpar}

\begin{mathpar}
\inferrule* [left=SelectDet]
    {\sigma \models b_1 = \bot \land \ldots \land
                    b_{i-1} = \bot \land
                    b_i = \top \land
                    b_{i+1} = \bot \land \ldots \land
                    b_n = \bot}
    {[b_1 \rightarrow P_1 \thickbar \ldots \thickbar
      b_i \rightarrow P_i \thickbar \ldots \thickbar
      b_n \rightarrow P_n], \sigma \rightarrow P_i, \sigma}

\inferrule* [left=SelectNonDet]
    {\sigma \models b_i = \top}
    {[b_1 \rightarrow P_1 \thickbar \ldots \thickbar
      b_i \rightarrow P_i \thickbar \ldots \thickbar
      b_n \rightarrow P_n], \sigma \rightarrow P_i, \sigma}
\end{mathpar}

\marginpar{avs: this is actually a little wrong}
\begin{mathpar}
\inferrule* [left=Repeat]
    {S, \sigma \rightarrow P, \sigma}
    {*S, \sigma \rightarrow P; *S, \sigma}

\inferrule* [left=RepeatNoneDet]
    {\sigma \models b_1 = \bot \land \ldots \land b_n = \bot}
    {*[b_1 \rightarrow P_1 \thickbar \ldots \thickbar
      b_n \rightarrow P_n], \sigma \rightarrow \Skip, \sigma}

\inferrule* [left=RepeatNoneNonDet]
    {\sigma \models b_1 = \bot \land \ldots \land b_n = \bot}
    {*[b_1 \rightarrow P_1 | \ldots |
      b_n \rightarrow P_n], \sigma \rightarrow \Skip, \sigma}
\end{mathpar}

\section{Static Analysis}

In the physical reality of implementing our language on hardware, if we ever
have two concurrent programs that write two different values to the same
variable simultaneously, we will connect voltage to ground through the variable,
which is a short circuit, and will cause the circuit to literally catch fire.
To prevent this from happening, we provide a static analysis method that tracks
which variables a program may write to, and prohibits parallel processes from
writing to the same variables.  Because HSE+ is Turing-complete modulo infinite
memory, we must take a worst-case evaluation of select statements.

Additionally, we track variables which a program may read from which are certain
to be not defined when they are read.  This does not provide strong guarantees,
since we could set a variable only in one branch of a guard before reading it,
but it is simple to track, and catches many programming errors.  When
implementing CHP in HSE+, it also allows us to provide a guarantee that at
least certain obvious cases of communication via non-channel shared variables is
not occurring.

A process or boolean has a \emph{footprint} assigned to it if it does not
contain the possibility of simultaneous writes.  A boolean's footprint is simply
the set of variables it reads from, and a process's footprint is a pair $(r,
w)$, where $w$ is the set of variables it writes to, and $r$ is the set of
variables that the process reads from before it could possibly have written to
them.

\subsection{Booleans}
\begin{mathpar}
\inferrule* [left=Primitive]
    { }
    {\ell : \emptyset}

\inferrule* [left=Read]
    { }
    {a : \{a\}}

\inferrule* [left=Not]
    {b : r}
    {\lnot b : r}

\inferrule* [left=And]
    {b_1 : r_1 \\ b_2 : r_2}
    {b_1 \land b_2 : r_1 \union r_2}

\inferrule* [left=Or]
    {b_1 : r_1 \\ b_2 : r_2}
    {b_1 \lor b_2 : r_1 \union r_2}
\end{mathpar}

\subsection{Programs}
\begin{mathpar}
\inferrule* [left=Skip]
    { }
    {skip : (\emptyset, \emptyset)}

\inferrule* [left=Write]
    {b : r}
    {a := b : (r, \{a\})}

\inferrule* [left=Sequence]
    {p_1:(r_1, w_1) \\ p_2:(r_2, w_2)}
    {p_1; p_2 : (r_1 \union (r_2 \setminus w_1), w_1 \union w_2)}

\inferrule* [left=Parallel]
    {p_1:(r_1, w_1) \\ p_2:(r_2, w_2) \\ w_1 \intersection w_2 = \emptyset}
    {p_1 || p_2 :
        ((r_1 \setminus w_2) \union (r_2\setminus w_1), w_1 \union w_2)}

\inferrule* [left=Selection]
    {b_1 : q_1, \ldots, b_n : q_n \\
     p_1 : (r_1, w_1), \ldots, p_n : (r_n, w_n)}
    {[b_1 \rightarrow p_1  | \ldots | b_n \rightarrow p_n] : 
     (\Union_{i=1}^{n} (q_i \union r_i), \Union_{i=1}^{n} w_i)}

\inferrule* [left=Repetition]
    {s : (r,w)}
    {\ast s : (r,w)}
\end{mathpar}

Let $p$ range over programs, and $s$ range over both deterministic and
non-deterministic selection statements.

The third requirement of the \textsc{Parallel} rule prevents fires; if the write
sets of both parallel programs do not overlap, then they cannot write to the
same variables, and we will not have short circuits.

\section{Crap about shared variables in CHP}

Track the set of \emph{vulnerable} variables (sum of free variables in all
booleans in programs).  Forbid parallel compositions in CHP where the set of
vulnerable variables and the set of written variables in the other process
overlap.

Also need a type inference system that will assign active/passive directionality
to channels

\end{document}
