\documentclass{article}
\usepackage{mathpartir}
\usepackage{amsmath}
\usepackage{amssymb}
\usepackage{stmaryrd}

\title{The $\star$ Operator}
\author{Stephen Longfield}

\begin{document}

\newcommand{\union}{\cup}
\newcommand{\Union}{\bigcup}
\newcommand{\intersection}{\cap}
\newcommand{\thickbar}{\talloblong}
\newcommand{\Skip}{\hbox{\bf skip}}

\maketitle

\section{Overview}

The $\star$ operator replaces the $\bullet$ operator in CHP. As far as I know, it was first suggested in Manohar01, and was done to allow for a more well-defined translation into HSE. The two differences between the operators are that $\bullet$ does not deadlock when composed with itself, and that while channel actions composed with $\bullet$ are defined to begin at the same time, channel actions composed with $\star$ are defined to begin and end at the same time. I think that it will vastly simplify our lives as compared to $\bullet$, as we no longer have to consider the parallel composition of bulleted together operatons. 

An additional restriction that we can place on the language is that probes can only occur in the guards of selections statements.  This means that any statements of the form $x := \bar{A}$ are illegal. I do not think this buys us anything at this level, but I used it at the HSE reshuffling level to simplify verification. This restriction doesn't really change anything unless we also outlaw negated probes (typically done at the PRS/HSE level, as the "falseness" of a probe is not usually stable), as $x := \bar{A}$ is equivalent to $[\bar{A} \rightarrow x\uparrow \talloblong \lnot \bar{A} \rightarrow x\downarrow]$

\section{Operational Semantics}

These aren't 100\% of the rules, I'm using the fact that that $||$ and $\star$ commute to not write down everything, and this doesn't handle situations of the form $A\star B||B \star C||A||C$ (note that if the last parallel composition was $A \star C$ it would deadlock).

\begin{mathpar}
\inferrule* [left=StarSend]
    {}
    {(A!\mathrm{b_1} \star B!\mathrm{b_2}) || A?x || B?y , \sigma \rightarrow \mathtt{skip}, \sigma[x = \mathrm{b_1}, y = \mathrm{b_2}]}

\inferrule* [left=StarRecv]
    {}
    {(A?x \star B?y) || A!\mathrm{b_1} || B!\mathrm{b_2} , \sigma \rightarrow \mathtt{skip}, \sigma[x = \mathrm{b_1}, y = \mathrm{b_2}]}

\inferrule* [left=StarSendRecv]
    {}
    {A?x \star B!\mathrm{b_2} || A!\mathrm{b_1} || B?y , \sigma \rightarrow \mathtt{skip}, \sigma[x = \mathrm{b_1}, y = \mathrm{b_2}]}

\end{mathpar}

An attempt at a more general rule, given that $C$ and $D$ are communication actions (of the form $A!\mathrm{b}$, $B?x$, or $C\star D$), and $C^\prime$ and $D^\prime$ are their complimentary actions.

\begin{mathpar}
\inferrule* [left=Star]
    {C || C^\prime, \sigma \rightarrow \mathtt{skip} \; \sigma^\prime \\
     D || D^\prime, \sigma^\prime \rightarrow \mathtt{skip} \; \sigma^{\prime\prime}}
    {C \star D || C^\prime || D^\prime, \sigma \rightarrow \mathtt{skip} \; \sigma^{\prime\prime}}
\end{mathpar}

I think this covers all of the cases, but I'm not 100\% certain.  I'd appreciate you taking a look at it.

\section{Translation into HSE+}

To simplify the translation, I'm going to propose we make the translation for send and receive the same, as it makes things a bit clearer. This requires a few definitions:

\begin{tabular}{| l | c | c | c |}
\hline
& $R!\top$ & $R!\perp$ & $R?x$ \\
\hline
$[r_i]$ & $[R_a]$ & $[R_a]$ & $[R_t \rightarrow x\uparrow \talloblong R_f \rightarrow x\downarrow]$ \\
\hline
$[\lnot r_i]$ & $ [\lnot R_a] $ & $[\lnot R_a]$ & $[\lnot R_t \land \lnot R_f ]$ \\
\hline
$r_o\uparrow$ & $R_t \uparrow $ & $R_f \uparrow$ & $R_a \uparrow$ \\
\hline
$r_o\downarrow$ & $R_t \downarrow $ & $R_f \downarrow$ & $R_a \downarrow$ \\
\hline
\end{tabular}


With that we now have:

\begin{align*}
\textrm{ $C$ and $D$ Passive: } & C \star D \overset{\triangle}{=} && [c_i \land r_i]; (l_o\uparrow || r_o\uparrow); [\lnot c_i \land \lnot r_i]; (l_o\downarrow || r_o\downarrow)   \\
\textrm{ $C$ Passive and $D$ Active: } & C \star D \overset{\triangle}{=}  && [c_i]; d_o\uparrow; [d_i]; c_o\uparrow; [\lnot c_i]; d_o\downarrow; [\lnot d_i]; c_o\downarrow\\
\textrm{ $C_0$ ... $C_n$ Passive and $D$ Active: } & C \star D \overset{\triangle}{=} &&  [c_{0_i} \land ... \land \; c_{n_i}]; d_o\uparrow; [d_i]; (c_{0_o}\uparrow || ... || c_{n_o}\uparrow); \\
& && [\lnot c_{0_i} \land ... \land \lnot c_{0_i} ]; d_o\downarrow; [\lnot d_i]; (c_{0_o} \downarrow || ... || c_{n_o}\downarrow)
\end{align*}

\section{Static Analysis Notes}

The rules for $\star$ are basically the same as $\bullet$, that we cannot have more than one active channel bulleted together. However, there's also the constraint that we can't have a cycle of bulleted together operations, or we will deadlock. This might be easy to check for the potential for? It's always nice to have a few of the potential deadlock conditions covered by static analysis, since deadlock's a bitch.

\end{document}
