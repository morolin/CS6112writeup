\documentclass{article}

\usepackage{amsmath}
\usepackage{amssymb}
\usepackage{stmaryrd}
\usepackage{fullpage}

\begin{document}

\title{CHP Translation and LTS Semantics}
\author{}
\maketitle

\section{CHP Translation}

\subsection{HSE+ Grammar}
\begin{align*}
    \ell & ::= \top \; | \bot \\
    \mathrm{b} & ::= a \; | \; \ell \; | \; \mathrm{b}_1 \wedge \mathrm{b}_2 \; | \; \mathrm{b}_1 \vee \mathrm{b}_2 \; | \; \lnot \mathrm{b} \\
    \mathrm{P} & ::= a := \mathrm{b} \; | \; S \; | \, *S \; | \; P_1; P_2 \; | \: P_1 || P_2 \; | \; \mathtt{skip} \\
    \mathrm{S} & ::= [ \mathrm{b}_1 \rightarrow P_1 \talloblong \; \ldots \; \talloblong \mathrm{b}_n \rightarrow P_n ] \; | \; [ \mathrm{b}_1 \rightarrow P_1 | \; \ldots \; | \mathrm{b}_n \rightarrow P_n ] \\
\end{align*}

We will use $\{P\}$ as shorthand to indicate terms which are translations of
channel actions in CHP.

\subsection{CHP Grammar}
\begin{align*}
    \ell & ::= \top \; | \bot \\
    \mathrm{b} & ::= a \; | \; \ell \; | \; \overline{A!} \; | \; \overline{A?} \; | \; \mathrm{b}_1 \wedge \mathrm{b}_2 \; | \; \mathrm{b}_1 \vee \mathrm{b}_2 \; | \; \lnot \mathrm{b} \\
    \mathrm{P} & ::= a := \mathrm{b} \; | \; S \; | \, *S \; | \; C \; | \; P_1; P_2 \; | \: P_1 || P_2 \; | \; \mathtt{skip} \\
    \mathrm{S} & ::= [ \mathrm{b}_1 \rightarrow P_1 \talloblong \; \ldots \; \talloblong \mathrm{b}_n \rightarrow P_n ] \; | \; [ \mathrm{b}_1 \rightarrow P_1 | \; \ldots \; | \mathrm{b}_n \rightarrow P_n ] \\
    \mathrm{C} & ::= A!\mathrm{b} \; | \; A?a \; | \; C_1 \bullet C_2
\end{align*}

The only terms appearing in this grammar but not the HSE+ grammar are:
$\overline{A!}$, $\overline{A?}$, $A!\mathrm{b}$, $A?a$, $C_1 \bullet C_2$.
Therefore, it suffices to give translations for just these extra terms;
everything else translates to its equivalent in HSE+. We also need to pick an
active end and a passive end for each channel in order to translate into HSE+.
We decide which end of the channel (the sending end or the receiving end) should
be active using static analysis, and denote by $\rho \vDash A!$ the channel $A$
having its sending end active, and $\rho \vDash A?$ the channel $A$ having its
receiving end active.  For simplicity, we restrict all channels to be 1-bit
channels. We also enforce channel end-to-endness and decide the scope of
channels (i.e. minimum enclosing scope outside of which no process ever
performs a send or receive on that channel).

\subsection{Translation}
\begin{align*}
    \overline{A!} & \Rightarrow (A_t \vee A_f) \\
    \overline{A?} & \Rightarrow A_a \\
    A!\mathrm{b} & \Rightarrow \left\{ \begin{matrix} [\mathrm{b} \rightarrow A_t := \top \talloblong \neg \mathrm{b} \rightarrow A_f := \top]; [A_a]; A_t := \bot; A_f := \bot; [\neg A_a] & \qquad \rho \vDash A! \\
                                                      [A_a]; [\mathrm{b} \rightarrow A_t := \top \talloblong \neg \mathrm{b} \rightarrow A_f := \top]; [\neg A_a]; A_t := \bot; A_f := \bot & \qquad \rho \vDash A? \end{matrix} \right. \\
    A?a & \Rightarrow \left\{ \begin{matrix} [A_t \vee A_f]; a := A_t; A_a := \top; [\neg A_t \wedge \neg A_f]; A_a := \bot & \qquad \rho \vDash A! \\
                                             A_a := \top; [A_t \vee A_f]; a := A_t; A_a := \bot; [\neg A_t \wedge \neg A_f] & \qquad \rho \vDash A? \end{matrix} \right. \\
     %%% Bulleted receive, both passive
    C_1?a \bullet C_2?a & \Rightarrow \left\{ \begin{matrix} [ ((C_{1_t} \vee C_{1_f}]; a := C_{1_t}; C_{1_a} := \top) || (C_{2_t} \vee C_{2_f}]; a := C_{2_t}; C_{2_a} := \top)); 
    \\ [\neg C_{1_t} \wedge \neg C_{1_f} \wedge \neg C_{2_t} \wedge \neg C_{2_f}]; (C_{1_a} := \bot\ ||C_{2_a} := \bot\ ) ]& \qquad \rho \vDash C_1!, C_2! \\
    %%% Bulleted receive, one active
     [ ((C_{1_t} \vee C_{1_f}]; a := C_{1_t}; C_{1_a} := \top) || (C_{2_t} \vee C_{2_f}]; a := C_{2_t}; C_{2_a} := \top)); 
    \\ [\neg C_{1_t} \wedge \neg C_{1_f}];  C_{2_a} := \bot\; [ \neg C_{2_t} \wedge \neg C_{2_f}]; C_{1_a} := \bot\ ]& \qquad \rho \vDash C_1!, C_2?
      \end{matrix} \right.  \\
      %%% Bulleted send, both passive
          C_1!b_1 \bullet C_2!b_2 & \Rightarrow \left\{ \begin{matrix} [
          ([C_{1_a}]; [\mathrm{b_1} \rightarrow C_{1_t} := \top \talloblong \neg \mathrm{b_1} \rightarrow C_{1_f} := \top]) || \\ ([C_{2_a}]; [\mathrm{b_2} \rightarrow C_{2_t} := \top \talloblong \neg \mathrm{b_2} \rightarrow C_{2_f} := \top])); 
    \\ [\neg C_{1_a} \wedge \neg C_{2_a}]; (C_{1_t} := \bot\ || C_{1_f} := \bot\ || C_{2_t} := \bot\ || C_{2_f} := \bot\ ) ]& \; \rho \vDash C_1?, C_2? \\
    %%% Bulleted send, one active
    ([C_{1_a}]; [\mathrm{b_1} \rightarrow C_{1_t} := \top \talloblong \neg \mathrm{b_1} \rightarrow C_{1_f} := \top]) || \\ ([\mathrm{b_2} \rightarrow C_{2_t} := \top \talloblong \neg \mathrm{b_2} \rightarrow C_{2_f} := \top]; [C_{2_a}])); 
    \\ [\neg C_{1_a}]; ( C_{2_t} := \bot\ || C_{2_f} := \bot\ );  [\neg C_{2_a}];  (C_{1_t} := \bot\ || C_{1_f} := \bot\ ) ] & \; \rho \vDash C_1?, C_2!
      \end{matrix} \right. \\
      %%% Bulleted send/recieve, both passive
          C_1!b \bullet C_2?a & \Rightarrow \left\{ \begin{matrix} [ ([C_{1_a}]; [\mathrm{b_1} \rightarrow C_{1_t} := \top \talloblong \neg \mathrm{b_1} \rightarrow C_{1_f} := \top]) || (C_{2_t} \vee C_{2_f}]; a := C_{2_t}; C_{2_a} := \top)); 
    \\ [\neg C_{1_a} \wedge \neg C_{2_t} \wedge \neg C_{2_f}]; (C_{1_t} := \bot\ || C_{1_f} := \bot\ ||C_{2_a} := \bot\ ) ]& \qquad \rho \vDash C_1!, C_2! \\
    %%% Bulleted send/recieve, receive active
         [ ([C_{1_a}]; [\mathrm{b_1} \rightarrow C_{1_t} := \top \talloblong \neg \mathrm{b_1} \rightarrow C_{1_f} := \top]) || C_{2_a} := \top; (C_{2_t} \vee C_{2_f}]; a := C_{2_t})); 
    \\ [\neg C_{1_a}]; C_{2_a} := \bot\ ; [ \neg C_{2_t} \wedge \neg C_{2_f}]; (C_{1_t} := \bot\ || C_{1_f} := \bot\ ) ]& \qquad \rho \vDash C_1!, C_2? \\
    %%% Builleted send/recieve, send active
      [ ( [\mathrm{b_1} \rightarrow C_{1_t} := \top \talloblong \neg \mathrm{b_1} \rightarrow C_{1_f} := \top];  [C_{1_a}] ) || (C_{2_t} \vee C_{2_f}]; a := C_{2_t}; C_{2_a} := \top)); 
    \\ [\neg C_{2_t} \wedge \neg C_{2_f}]; (C_{1_t} := \bot\ || C_{1_f} := \bot\ ); [\neg C_{1_a}]; C_{2_a} := \bot\ ) ]& \qquad \rho \vDash C_1?, C_2!
      \end{matrix} \right. 
\end{align*}

Here, $A_a, A_t, A_f$ are three new variables that we create for the purposes of
translating actions on channel $A$ into HSE+.

Above, the probe syntax only details when you have two channels combined
together with a bullet.  If you are combining any more, they must be passive.
Their first two actions (a wait and a set) will execute in parallel with all
others, then the second wait will be put into the combined AND wait, and their
set will be done in parallel.

We will use $\{A!\mathrm{b}\}$ and $\{A?a\}$ as shorthand for either translation
of $A!\mathrm{b}$ and $A?a$, respectively.

\section{LTS Semantics}
We begin by defining a ternary relation and a function. First, we define
$\mathsf{match} \subseteq \mathbf{Labels} \times \mathbf{Labels} \times
\mathbf{Assignments}$ as follows:
$$
    \frac{A \in \mathbf{Channels} \quad a \in \mathbf{Variables}}{(\{A!\ell\}, \{A?a : \ell\}, \{[a = \ell]\}) \in \mathsf{match}} \qquad
    \frac{(\delta, \delta', \Sigma) \in \mathsf{match}}{(\delta', \delta, \Sigma) \in \mathsf{match}} $$$$
    \frac{(\delta, \delta', \Sigma) \in \mathsf{match}}{(\delta \cup \{A!\ell\}, \delta' \cup \{A?a : \ell\}, \Sigma \cup \{[a = \ell\}) \in \mathsf{match}}
$$
TODO
We do not explicitly forbid things like $A!\top \bullet A?a$ and $A?a \bullet
B?a$ here; perhaps we should? The second one will be caught by static checking,
at least (since it causes fires\ldots).

Next, we define $\mathsf{head}: \mathbf{Programs} \rightarrow
2^{\mathbf{ChannelActions}}$ as follows:
\begin{flalign*}
    \mathsf{head}(P || Q) & = \mathsf{head}(P) \cup \mathsf{head}(Q) \\
    \mathsf{head}(P ; Q) & = \mathsf{head}(P) \\
    \mathsf{head}(A!\ell) & = \{A!\} \\
    \mathsf{head}(A?a) & = \{A?\} \\
    \mathsf{head}(A!\ell \bullet C) & = \{A!\} \cup \mathsf{head}(C) \\
    \mathsf{head}(A?a \bullet C) & = \{A?\} \cup \mathsf{head}(C) \\
    \mathsf{head}(P) & = \varnothing \quad \text{for any other program $P$}
\end{flalign*}

We start with the boolean reduction rules. Here, $P, \sigma$ are the enclosing
program $P$ and the current environment $\sigma$, respectively.
$$
    \frac{}{P, \sigma \models \ell = \ell} \qquad
    \frac{[a = \ell] \in \sigma}{P, \sigma \models a = \ell} \qquad
    \frac{\sigma \models \mathrm{b} = \ell}{\sigma \models \neg \mathrm{b} = \neg \ell} $$$$
    \frac{\sigma \models \mathrm{b} = \top}{\sigma \models \mathrm{b} \vee \mathrm{b'} = \top} \qquad
    \frac{\sigma \models \mathrm{b'} = \top}{\sigma \models \mathrm{b} \vee \mathrm{b'} = \top} \qquad
    \frac{\sigma \models \mathrm{b} = \bot \quad \sigma \models \mathrm{b'} = \bot}{\sigma \models \mathrm{b} \vee \mathrm{b'} = \bot} $$$$
    \frac{\sigma \models \mathrm{b} = \bot}{\sigma \models \mathrm{b} \wedge \mathrm{b'} = \bot} \qquad
    \frac{\sigma \models \mathrm{b'} = \bot}{\sigma \models \mathrm{b} \wedge \mathrm{b'} = \bot} \qquad
    \frac{\sigma \models \mathrm{b} = \top \quad \sigma \models \mathrm{b'} = \top}{\sigma \models \mathrm{b} \wedge \mathrm{b'} = \top} $$$$
    \frac{A! \in \mathsf{head}(P)}{P, \sigma \models \overline{A!} = \top} \qquad
    \frac{A! \notin \mathsf{head}(P)}{P, \sigma \models \overline{A!} = \bot} \qquad
    \frac{A? \in \mathsf{head}(P)}{P, \sigma \models \overline{A?} = \top} \qquad
    \frac{A? \notin \mathsf{head}(P)}{P, \sigma \models \overline{A?} = \bot}
$$

TODO use the new boolean notation? $P, \sigma \models$ instead of just $\sigma \models$\\
Here are the rules for CHP:
$$
    % skip, assign, sequential and parallel composition rules
    \frac{\langle P, \sigma \rangle \xrightarrow{\delta} \langle P', \sigma' \rangle}{\langle P ; Q, \sigma \rangle \xrightarrow{\delta} \langle P' ; Q, \sigma' \rangle} \qquad
    \frac{\langle P, \sigma \rangle \xrightarrow{\delta} \langle P', \sigma' \rangle}{\langle P || Q, \sigma \rangle \xrightarrow{\delta} \langle P' || Q, \sigma' \rangle} \qquad
    \frac{\langle Q, \sigma \rangle \xrightarrow{\delta} \langle Q', \sigma' \rangle}{\langle P || Q, \sigma \rangle \xrightarrow{\delta} \langle P || Q', \sigma' \rangle} \qquad
    \frac{}{\langle \mathtt{skip}; P, \sigma \rangle \xrightarrow{\tau} \langle P, \sigma \rangle} $$$$
    \frac{\sigma \models \mathrm{b} = \ell}{\langle a := \mathrm{b}, \sigma \rangle \xrightarrow{\tau} \langle \mathtt{skip}, \sigma[a = \ell] \rangle} \qquad
    \frac{}{\langle \mathtt{skip} || P, \sigma \rangle \xrightarrow{\tau} \langle P, \sigma \rangle} \qquad
    \frac{}{\langle P || \mathtt{skip}, \sigma \rangle \xrightarrow{\tau} \langle P, \sigma \rangle} $$$$
    % selection rules
    \frac{\sigma \models \mathrm{b}_1 = \bot \wedge \; \ldots \; \wedge \; \mathrm{b}_{i-1} = \bot \; \wedge \mathrm{b}_i = \top \wedge \mathrm{b}_{i+1} = \bot \wedge \; \ldots \; \wedge \; \mathrm{b}_n = \bot} {\langle [ \mathrm{b}_1 \rightarrow P_1  \talloblong \; \ldots \; \talloblong \mathrm{b}_n \rightarrow P_n ] , \sigma \rangle \xrightarrow{\tau} \langle P_i , \sigma\rangle  } \qquad
    \frac{\sigma \models \mathrm{b}_i = \top} {\langle [ \mathrm{b}_1 \rightarrow P_1  | \; \ldots \; | \mathrm{b}_n \rightarrow P_n ] , \sigma \rangle \xrightarrow{\tau} \langle P_i , \sigma\rangle  } $$$$
    \frac{\sigma \models b_1 = \bot \wedge \ldots \wedge b_n = \bot}{\langle *[ \mathrm{b}_1 \rightarrow P_1  \talloblong \; \ldots \; \talloblong \mathrm{b}_n \rightarrow P_n ] , \sigma \rangle \xrightarrow{\tau} \langle \mathtt{skip} , \sigma\rangle } \qquad
    \frac{\sigma \models b_1 = \bot \wedge \ldots \wedge b_n = \bot}{\langle *[ \mathrm{b}_1 \rightarrow P_1  | \; \ldots \; | \mathrm{b}_n \rightarrow P_n ] , \sigma \rangle \xrightarrow{\tau} \langle \mathtt{skip} , \sigma \rangle } $$$$
%   \frac{\langle S, \sigma \rangle \xrightarrow{\tau} \langle P, \sigma \rangle}{\langle *S, \sigma \rangle \xrightarrow{\tau} \langle P;*S, \sigma \rangle} $$$$
    \frac{\sigma \models \mathrm{b}_1 = \bot \wedge \; \ldots \; \wedge \; \mathrm{b}_{i-1} = \bot \wedge \mathrm{b}_i = \top \wedge \mathrm{b}_{i+1} = \bot \wedge \; \ldots \; \wedge \; \mathrm{b}_n = \bot} {\langle *[ \mathrm{b}_1 \rightarrow P_1  \talloblong \; \ldots \; \talloblong \mathrm{b}_n \rightarrow P_n ] , \sigma \rangle \xrightarrow{\tau} \langle P_i; \; *[ \mathrm{b}_1 \rightarrow P_1  \talloblong \; \ldots \; \talloblong \mathrm{b}_n \rightarrow P_n ] , \sigma\rangle  } $$$$
    \frac{\sigma \models \mathrm{b}_i = \top} {\langle *[ \mathrm{b}_1 \rightarrow P_1  | \; \ldots \; | \mathrm{b}_n \rightarrow P_n ] , \sigma \rangle \xrightarrow{\tau} \langle P_i;\;*[ \mathrm{b}_1 \rightarrow P_1  | \; \ldots \; | \mathrm{b}_n \rightarrow P_n ] , \sigma\rangle  } $$$$
    % communication rules
    \frac{\langle P, \sigma \rangle \xrightarrow{\delta} \langle P', \sigma' \rangle \quad A!\ell \not \in \delta \quad A?a:\ell \not \in \delta}{\langle P, \sigma \rangle \xrightarrow{\delta} \langle P', \sigma' \rangle} $$$$
    \frac{\sigma \models \mathrm{b} = \top}{\langle \{A!\mathrm{b}\}, \sigma \rangle \xrightarrow{A!\top} \langle \mathtt{skip}, \sigma[A_t = \top, A_a = \top] \rangle} \qquad
    \frac{\sigma \models \mathrm{b} = \bot}{\langle \{A!\mathrm{b}\}, \sigma \rangle \xrightarrow{A!\bot} \langle \mathtt{skip}, \sigma[A_f = \top, A_a = \top] \rangle} $$$$
    \frac{}{\langle \{A?a\}, \sigma \rangle \xrightarrow{A?a : \top} \langle \mathtt{skip}, \sigma[A_t = \top, A_a = \top, a = \top] \rangle} \qquad
    \frac{}{\langle \{A?a\}, \sigma \rangle \xrightarrow{A?a : \bot} \langle \mathtt{skip}, \sigma[A_f = \top, A_a = \top, a = \bot] \rangle} $$$$
    %\frac{\langle P, \sigma \rangle \xrightarrow{A!\ell} \langle P', \sigma \rangle \quad \langle Q, \sigma' \rangle \xrightarrow{A?a : \ell} \langle Q', \sigma'' \rangle}{\langle P || Q, \sigma \rangle \xrightarrow{\tau} \langle P' || Q', \sigma[a = \ell] \rangle} $$$$
    %\frac{\langle P, \sigma \rangle \xrightarrow{A?a : \ell} \langle P', \sigma' \rangle \quad \langle Q, \sigma'' \rangle \xrightarrow{A!\ell} \langle Q', \sigma'' \rangle}{\langle P || Q, \sigma \rangle \xrightarrow{\tau} \langle P' || Q', \sigma[a = \ell] \rangle}
    \frac{\langle P, \sigma \rangle \xrightarrow{\delta} \langle P', \sigma' \rangle \quad \langle Q, \sigma'' \rangle \xrightarrow{\delta'} \langle Q', \sigma''' \rangle \quad (\delta, \delta', \Sigma) \in \mathsf{match}}{\langle P || Q, \sigma \rangle \xrightarrow{\tau} \langle P' || Q', \sigma[\Sigma] \rangle} $$$$
    % bullet rules
    \frac{\langle \{C\}, \sigma \rangle \xrightarrow{\delta} \langle \mathtt{skip}, \sigma' \rangle \quad \sigma \models \mathrm{b} = \top}{\langle \{A!\top \bullet C\}, \sigma \rangle \xrightarrow{\delta \cup \{A!\top\}} \langle \mathtt{skip}, \sigma'[A_t = \top, A_a = \top] \rangle} $$$$
    \frac{\langle \{C\}, \sigma \rangle \xrightarrow{\delta} \langle \mathtt{skip}, \sigma' \rangle \quad \sigma \models \mathrm{b} = \bot}{\langle \{A!\bot \bullet C\}, \sigma \rangle \xrightarrow{\delta \cup \{A!\bot\}} \langle \mathtt{skip}, \sigma'[A_f = \top, A_a = \top] \rangle} $$$$
    \frac{\langle \{C\}, \sigma \rangle \xrightarrow{\delta} \langle \mathtt{skip}, \sigma' \rangle}{\langle \{A?a \bullet C\}, \sigma \rangle \xrightarrow{\delta \cup \{A?a : \top\}} \langle \mathtt{skip}, \sigma'[A_t = \top, A_a = \top, a = \top] \rangle} $$$$
    \frac{\langle \{C\}, \sigma \rangle \xrightarrow{\delta} \langle \mathtt{skip}, \sigma' \rangle}{\langle \{A?a \bullet C\}, \sigma \rangle \xrightarrow{\delta \cup \{A?a : \bot\}} \langle \mathtt{skip}, \sigma'[A_f = \top, A_a = \top, a = \top] \rangle} $$$$
$$

$\sigma$ is the environment, which in this case consists of a set of mappings of variables to values.
We initialize $\sigma$ to include $A_a = \bot$, $A_t = \bot$, $A_f = \bot$ for all channels $A$.
Furthermore, note that the labels on the transitions are sets, since we can have several channel operations occurring at once (via $\bullet$).

TODO: probes are currently broken, and will always evaluate to false (since they are only `true' when the other process has performed a labeled transition, but before the react rule takes place; we can never actually be in this state. Should we use the old $\rho$ method to make probes work again?)
We fixed this using some new boolean rules, but still need to come up with a good notation and add them to this document.

\section{Proof of Correctness}
We want to show that the transitions of the LTS correspond to those of the HSE+.
First, we start by giving a definition of \textbf{channel} and \textbf{channel action}.

A \textbf{channel} is used for inter-process communication. All channels are
uniquely identified using a capital letter, and there are two basic
\textbf{channel actions} allowed on channels: sends (denoted using
$A!\mathrm{b}$, which sends the boolean value b on channel $A$) and receives
(denoted using $A?a$, which receives a sent value from channel $A$ and stores it
in variable $a$). Sends and receives happen simultaneously; if a process
attempts to send on a channel, it blocks until another process attempts to
receive on that channel, at which point both actions occur simultaneously.
Likewise, if a process attempts to receive on a channel, it blocks until another
process attempts to send on that channel, at which point both actions occur
simultaneously. Channel actions are program actions.

Channels are \textbf{end-to-end}, which means that at most one process can ever
send on any given channel, and at most one process can ever receive on any given
channel (where ``process'' is defined by parallel composition: any two programs
parallel composed with each other are treated as two separate processes).

We also define \textbf{probes} on channels. The two probes are $\overline{A!}$
and $\overline{A?}$; the former evaluates to $\top$ if a process is currently
attempting to send on channel $A$, and $\bot$ otherwise, and the latter
evaluates to $\top$ if a process is currently attempting to receive on channel
$A$, and $\bot$ otherwise. Probes are boolean expressions, and in addition are
exempt from the end-to-end restriction.

Finally, we define the \textbf{bullet} operator. This operator operates on
channel actions, and its semantics are as follows: for any two channel actions
$C$ and $C'$, $C \bullet C'$ is a single channel action that completes when both
$C$ and $C'$ complete simultaneously. So long as either one is blocked, $C
\bullet C'$ is also blocked. Thus \textbf{bullet} enforces simultaneity.

\vspace{0.1in}

We begin the proof of correspondence by noting that for anything aside from
probes and channel actions, since the transition rules are identical, it follows
that the behavior is also identical. Thus we only need to show that the
translations of channel actions and probes correspond to the definition above.
Incidentally, due to the details of the hardware implementation, there are many
programs that are valid in CHP, yet will not pass static analysis and thus
cannot be translated into HSE+. We seek to prove the correctness of the
translation only for programs that are both valid CHP and pass the static
analyses.

The translation of the two basic channel actions is the ``four-phase
handshake'', which works as follows: every channel has three variables assigned
to it. For a channel $A$, our translation rules name these as $A_a$, $A_t$, and
$A_f$. Initially (i.e. when no channel action is being performed), all three of
these variables are set to false. In addition, for each channel, we must pick
one of the ends to be the active end, and the other to be the passive end. Thus
there are two cases.

If the sending end of the channel is chosen to be the passive end, then the
four-phase handshake proceeds as follows: the receiving end of the channel,
being the active end, initiates the handshake by setting $A_a$ to be true.  It
then waits for data to be sent by checking for either $A_t$ or $A_f$ to be true.
The sending end, being the passive end, waits for the handshake to be initiated
by checking for $A_a$ to be true. It then initiates the send by setting either
$A_t$ or $A_f$ to true, depending on which value is to be sent. The receiving
end, upon receiving the data and setting $a$ to the appropriate value, indicates
receipt by setting $A_a$ to be false; it then waits for the sending end to
acknowledge the end of the handshake by checking for either $A_t$ or $A_f$ to be
false again (at which point all three variables will have returned to their
original, false states). The sending end, upon seeing the acknowledgement of
receipt, sets $A_t$ and $A_f$ to be false again, thus ending the handshake. This
handshake enforces simultaneity: if a process attempts to send without there
being a receiver, then it will block until another process attempts to receive
and sets $A_a$ to true. If a process attempts to receive without there being a
sender, then it will set $A_a$ to true, but then block until another process
sends a value by setting either $A_t$ or $A_f$ to be true. Finally, once the
handshake is complete, $A_a$, $A_t$, and $A_f$ are all set to false once more,
resetting the channel in preparation for another channel action.

The translations for the case where the sending end is passive (i.e. $\rho
\vDash A?$) are as follows:
\begin{flalign*}
    A!\mathrm{b} & \Rightarrow [A_a]; [\mathrm{b} \rightarrow A_t := \top \talloblong \neg \mathrm{b} \rightarrow A_f := \top]; [\neg A_a]; A_t := \bot; A_f := \bot \\
    A?a & \Rightarrow A_a := \top; [A_t \vee A_f]; a := A_t; A_a := \bot; [\neg A_t \wedge \neg A_f]
\end{flalign*}
As can be seen, they exactly correspond to the description given above. Thus,
the translations are correct in this case.

If the receiving end of the channel is chosen to be the passive end, then the
four-phase handshake proceeds as follows: the sending end of the channel,
being the active end, initiates the handshake by setting either $A_t$ or $A_f$
to be true, depending on which value is to be sent. It then waits for
acknowledgement of receipt by checking for $A_a$ to be true. The receiving end,
being the passive end, waits for the handshake to be initiated by checking for
either $A_t$ or $A_f$ to be true. It then assigns $a$ to the appropriate value
and acknowledges receipt by settings $A_a$ to be true. The sending end, upon
receiving acknowledgement, indicates this by setting $A_t$ or $A_f$ to be false
again, and waits for acknowledgement of termination of the handshake by waiting
for $A_a$ to be false. Finally, the receiving end, upon seeing that $A_t$ or
$A_f$ is false again, sets $A_a$ to be false, thus ending the handshake. This
handshake enforces simultaneity: if a process attempts to send without there
being a receiver, then it will set either $A_t$ or $A_f$ to true, but then block
until another process acknowledges receipt by setting $A_a$. If a process
attempts to receive without there being a sender, then it will block until
another process attempts to send and sets either $A_t$ or $A_f$ to true.
Finally, once the handshake is complete, $A_a$, $A_t$, and $A_f$ are all set to
false once more, resetting the channel in preparation for another channel
action.

The translations for the case where the receiving end is passive (i.e. $\rho
\vDash A!$) are as follows:
\begin{flalign*}
    A!\mathrm{b} & \Rightarrow [\mathrm{b} \rightarrow A_t := \top \talloblong \neg \mathrm{b} \rightarrow A_f := \top]; [A_a]; A_t := \bot; A_f := \bot; [\neg A_a] \\
    A?a & \Rightarrow [A_t \vee A_f]; a := A_t; A_a := \top; [\neg A_t \wedge \neg A_f]; A_a := \bot
\end{flalign*}
As can be seen, they exactly correspond to the description given above. Thus,
the translations are correct in this case.

\vspace{0.05in}

In the case of bullet, to keep things simple, we will only look at dataless
channels. The channels can easily be extended to send and receive data by
replacing one or both of the channel communication variables with a set of
variables. Since we are only dealing with dataless channels, we will use $A!$ to
indicate communication on a channel $A$ for either end (as the ends are now
equivalent); we also use $A_a$ and $A_b$ as the two communication variables for
$A$.

In order to enforce simultaneity, we must place some restrictions on the set of
channels in the bullet; to be specific, we can allow at most one channel in the
set to be active, and the rest must be passive. In the case of $n$ passive
channels $A_1, \ldots, A_n$, the translation for $A_1! \bullet \ldots \bullet
A_n!$ is:
$$[{A_1}_a \wedge \ldots \wedge {A_n}_a]; {A_1}_b := \top; \ldots; {A_n}_b := \top; [\neg {A_1}_a \wedge \ldots \wedge \neg {A_n}_a]; {A_1}_b := \bot; \ldots; {A_n}_b := \bot$$
In the case of one active channel $B$ and $n$ passive channels $A_1, \ldots,
A_n$, the translation for $A_1! \bullet \ldots \bullet A_i! \bullet B \bullet
A_{i + 1}! \bullet \ldots \bullet A_n!$ is:
$$B_a := \top; [B_b \wedge {A_1}_a \wedge \ldots \wedge {A_n}_a]; {A_1}_b := \top; \ldots; {A_n}_b := \top; [\neg {A_1}_a \wedge \ldots \wedge \neg {A_n}_a]; B_a := \bot; [\neg B_b]; {A_1}_b := \bot; \ldots; {A_n}_b := \bot$$

We will start with the first case, when all of the channels are passive.
Clearly, the translation implements the four-phase handshake: for each channel
$A_i$, we first wait for ${A_i}_a$ to go high, then we set ${A_i}_b$ to high and
wait for ${A_i}_a$ to go low, and finally we set ${A_i}_b$ to low. We also
perform the last step simultaneously for all channels, thus fitting the
definition of bullet.

For the second case, we complete all of the passive channel actions correctly
and simultaneously for the same reasons as in the first case. For channel $B$,
we again implement the four-phase handshake: we first set $B_a$ to high, then we
wait for $B_b$ to go high, then we set $B_a$ to low, and finally we wait for
$B_b$ to go low. We wait until $B_b$ is low before setting each of the ${A_i}_b$
to low, thus completing all channel actions simultaneously.

\vspace{0.05in}

Next, we prove that the translations of the probes are correct. Due to
limitations of the hardware implementation, we can only ever probe the active
end of the channel, since the passive end makes no visible changes when it is
blocked on a channel action. Thus, for a channel with a passive receiving end,
the only probe allowed is a send probe ($\overline{A!}$); likewise, for a
channel with a passive sending end, the only probe allowed is a receive probe
($\overline{A?}$). Thus we only have one translation for each probe. The
translations are as follows:
\begin{flalign*}
    \overline{A!} & \Rightarrow (A_t \vee A_f) \\
    \overline{A?} & \Rightarrow A_a \\
\end{flalign*}
The send probe will only ever be true at the very beginning of an active send:
when a process attempts to send (by setting $A_t$ or $A_f$ to true). Likewise,
the receive probe will only ever be true at the very beginning of an active
receive: when a process attempts to receive (by setting $A_a$ to true). This
behaviour corresponds to the definition of probe; thus, the translations are
correct in the cases where they apply.

TODO: proof for bullet here

\end{document}
