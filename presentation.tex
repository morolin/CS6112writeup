\documentclass[compress]{beamer}
\newcommand{\navwidth}{\paperwidth}
\pgfdeclareimage[height=0.03\paperheight]{botlogo}{csllogo.pdf}
%Fiddling around with the Beamer template
\setbeamertemplate{headline}
{%
  \begin{beamercolorbox}[colsep=1.5pt]{upper separation line head}
  \end{beamercolorbox}
  \begin{beamercolorbox}{section in head/foot}
    \vskip2pt\insertnavigation{\navwidth}\vskip2pt
  \end{beamercolorbox}%
}
\setbeamertemplate{navigation symbols}{}
\setbeamertemplate{footline}
{%
\vskip2pt%
\makebox[\paperwidth]{\hskip2pt%
\hskip2pt\insertsection:\insertsubsection\hfill%
\insertframenumber/\inserttotalframenumber\hskip2pt}%
\vskip2pt
}
\usepackage{times}  % fonts are up to you
\usepackage{amsmath}
\usepackage{amssymb}
\usepackage{stmaryrd}
\usepackage{xcolor}
% these will be used later in the title page
\title{Formalizing Communicating Hardware Processes}
\author{Stephen Longfield \\
Alec Story \\
Norris Xu
\date{\today}
}
% have this if you'd like a recurring outline
%\AtBeginSection[]  % "Beamer, do the following at the start of every section"
%{
%\begin{frame}<beamer>
%\frametitle{Outline} % make a frame titled "Outline"
%\tableofcontents[currentsection]  % show TOC and highlight current section
%\end{frame}
%}
\begin{document}
% this prints title, author etc. info from above
\begin{frame}[plain]
\titlepage
\end{frame}
\section{Introduction}
\subsection{Overview}
\begin{frame}
\frametitle{Overview}
\begin{itemize}
\item Verification of hardware systems
\item Asynchronous computing
\begin{itemize}
\item Manufacturability concerns
\item Formal specifications
\end{itemize}
\item Concurrency-increasing transformations
\end{itemize}
\end{frame}
\subsection{Motivation}
\begin{frame}
\frametitle{Motivation of verification}
\begin{itemize}
\item Verified hardware is always nice
\item Manufacturing for hardware is very expensive
\begin{itemize}
\item Hardware compilation and test cycles cost millions of dollars
\item Unit testing and verification is difficult
\item May miss corner cases
\end{itemize}
\item Verification is very difficult
\begin{itemize}
\item Stochastic effects of temperature and manufacturing variation
\end{itemize}
\end{itemize}
\end{frame}
\subsection{Asynchronous Computing}
\begin{frame}
\frametitle{Asynchronous Computing}
\begin{itemize}
\item Circuit does not make any timing guarantees
\begin{itemize}
\item no longer required to design to a clock
\end{itemize}
\item Explicit synchronization
\item These make it possible to verify in ways impossible for synchronous circuits
\end{itemize}
\end{frame}
\section{Language Overview}
\subsection{Base Language}
\begin{frame}
\frametitle{Circuit Description Langauge}
\begin{itemize}
\item Describe a language of parallel processes with shared variables
\begin{itemize}
\item Closer to hardware implementation
\item Easy to describe a variety of circuits
\end{itemize}
\item On top of it build a message-passing concurrency language
\begin{itemize}
\item Restrict to the implementable set
\end{itemize}
\end{itemize}
\end{frame}
\begin{frame}
\frametitle{Simple statements}
\begin{itemize}
\item Null statement: $\texttt{skip}$, which does nothing
\item Variable set: $x := \mathrm{b}$
\end{itemize}
\end{frame}
\begin{frame}
\begin{itemize}
\item Boolean:
\begin{itemize}
\item Literals: $\texttt{true}$ and $\texttt{false}$
\item Variables: $x$
\item AND: $\mathrm{b_1} \land \mathrm{b_2}$
\item OR: $\mathrm{b_1} \lor \mathrm{b_2}$
\item NOT: $\lnot \mathrm{b}$
\end{itemize}
\end{itemize}
\end{frame}
\begin{frame}
\frametitle{Program Composition}
\begin{itemize}
\item Sequential: $P_1; P_2$
\item Parallel: $P_1 || P_2$
\end{itemize}
\end{frame}
\begin{frame}
\frametitle{Selection Statements}
\begin{itemize}
\item Deterministic guarded selection
\begin{itemize}
\item $[\mathrm{b_1} \rightarrow P_1 \talloblong ... \talloblong \mathrm{b_n} \rightarrow P_n]$
\item Execute the statement $P_i$ whose guard, $\mathrm{b_i}$ is true
\item Guards must be mutually exclusive
\end{itemize}
\pause
\item Nondeterministic guarded selection
\begin{itemize}
\item $ [\textrm{b}_1 \rightarrow P_1 | \ldots | \textrm{b}_1 \rightarrow P_n ] $
\item Guards are not required to be mutually exclusive
\item Assumes demonic nondeterminism
\end{itemize}
\end{itemize}
\end{frame}
\begin{frame}
\frametitle{Repeated Selection Statements}
\begin{itemize}
\item  $*[\textrm{b}_1 \rightarrow P_1 \talloblong ... \talloblong \textrm{b} \rightarrow P_n]$
\begin{itemize}
\item As long as one guard is true, execute its corresponding statement
\item Guards must be mutually exclusive
\end{itemize}
\pause
\item Nondeterminstic guarded selection
\begin{itemize}
\item Guards do not need to be mutually exclusive
\end{itemize}
\end{itemize}
\end{frame}
\subsection{CHP Language}
\begin{frame}
\frametitle{Communication Actions}
\begin{itemize}
\item Would like to encode a CSP-like language on top of the shared variable framework
\pause
\item Encode communication actions in a four-phase variable handshake
\end{itemize}
\end{frame}
\begin{frame}
\frametitle{Value send: $X!\textrm{b}$}
\begin{itemize}
\item Active:
\begin{align*}
    & [\textrm{b} \rightarrow X_t := \texttt{true} \talloblong
     \lnot \textrm{b} \rightarrow X_f := \texttt{true}]; \\
    & \hspace{.5cm} [X_a]; (X_t := \texttt{false} || X_f := \texttt{false}); [\lnot X_a]
\end{align*}
\item Passive:
\begin{align*}
    & [X_a]; [\textrm{b} \rightarrow X_t := \texttt{true} \talloblong
       \lnot \textrm{b} \rightarrow X_f := \texttt{true} ]; \\
    & \hspace{.5cm} [\lnot X_a]; (X_t := \texttt{false} || X_f := \texttt{false})
\end{align*}
\end{itemize}
\end{frame}
\begin{frame}
\frametitle{Variable receive: $X?y$}
\begin{itemize}
\item Active:
\begin{align*}
    & X_a := \texttt{true};
     [X_t \rightarrow y := \texttt{true} \talloblong
      X_f \rightarrow y := \texttt{false}]; \\
    & \hspace{.5cm} X_a := \texttt{false};
      [\lnot X_t \land \lnot X_f]
\end{align*}
\item Passive:
\begin{align*}
    & [X_t \rightarrow y := \texttt{true} \talloblong
       X_f \rightarrow y := \texttt{false}]; \\
    & \hspace{.5cm} X_a := \texttt{true};
      [\lnot X_t \land \lnot X_f];
      X_a := \texttt{false};
\end{align*}
\end{itemize}
\end{frame}
\begin{frame}
\frametitle{Channel probe: $\overline{C?}$ or $\overline{C!}$}
\begin{itemize}
\item True when that channel is ready to send or receive.
\item $\overline{C?} \mapsto X_a$
\item $\overline{C!} \mapsto (X_t \lor X_f)$
\end{itemize}
\end{frame}
\begin{frame}
\frametitle{Simultaneous Composition}
\begin{itemize}
\item Channel actions can synchronize between two processes
\item It may be the case that we want to synchronize between several processes
\item In that case, use simultaneous composition of channel actions:
\begin{itemize}
\item $C \bullet D$ where both $C$ and $D$ are some channel action
\end{itemize}
\end{itemize}
\end{frame}
\section{Implementability Restrictions}
\subsection{Channel Directionality}
\begin{frame}
\frametitle{Channel Directionality}
\pause
\begin{itemize}
    \item Before we can implement, channels need direction
    \pause
    \item If there are no restrictions, we can choose them arbitrarily
\end{itemize}
\end{frame}
\begin{frame}
\frametitle{Restrictions}
\pause
\begin{itemize}
    \item Only probes on the active end of a channel are implementable
    \pause
    \item Bullets are messy
    \pause
    \begin{itemize}
        \item Synchronization makes implementation hard
        \pause
        \item Only one of the actions in a bullet can be on the active end of a channel, the rest must be passive
    \end{itemize}
\end{itemize}
\end{frame}
\begin{frame}[plain]
\begin{center}
\huge 2-Sat!
\end{center}
\end{frame}
\subsection{Concurent Writes}
\begin{frame}
\frametitle{What's wrong with this program?}
\begin{center}
    \begin{displaymath}
    x := \texttt{true} || x := \texttt{false}
    \end{displaymath}
    \pause
    \LARGE It catches fire!
\end{center}
\end{frame}
\begin{frame}
\frametitle{How about this one?}
\begin{center}
    \begin{displaymath}
    A?x \bullet B?x
    \end{displaymath}
    \pause
    \LARGE It also catches fire!
\end{center}
\end{frame}
\begin{frame}
\frametitle{Detecting Potential Fires}
\begin{itemize}
\item The only threat occurs at parallel composition, and in bullets
\pause
\item Just forbid parallel compositions where the set of variables written to on one side is not disjoint from the set of variables on the other side
\pause
\item Also, check bullets to ensure that no variable or channel appears more than once
\pause
\item This also enforces end-to-end-ness of channels.
\end{itemize}
\end{frame}
\section{Formal Semantics and LTS}
\subsection{CHP Grammar}
\begin{frame}
    \frametitle{CHP Grammar}
    \begin{align*}
        \ell & ::= \mathtt{true} \; | \mathtt{false} \\
        \mathrm{b} & ::= a \; | \; \ell \; | \; {\color{red} \overline{A!}} \; | \; {\color{red} \overline{A?}} \; | \; \mathrm{b}_1 \wedge \mathrm{b}_2 \; | \; \mathrm{b}_1 \vee \mathrm{b}_2 \; | \; \lnot \mathrm{b} \\
        \mathrm{P} & ::= a := \mathrm{b} \; | \; S \; | \, *S \; | \; {\color{red} C} \; | \; P_1; P_2 \; | \: P_1 || P_2 \; | \; \mathtt{skip} \\
        \mathrm{S} & ::= [ \mathrm{b}_1 \rightarrow P_1 \talloblong \; ... \; \talloblong \mathrm{b}_n \rightarrow P_n ] \; | \; [ \mathrm{b}_1 \rightarrow P_1 | \; ... \; | \mathrm{b}_n \rightarrow P_n ] \\
        {\color{red} \mathrm{C}} & {\color{red} ::= A!\mathrm{b} \; | \; A?a \; | \; C_1 \bullet C_2}
    \end{align*}
\end{frame}
\begin{frame}
    New to CHP:
    \begin{itemize}
        \item Channels $A$
        \item Channel actions $A!\mathrm{b}, A?a$
        \item Channel probes $\overline{A!}, \overline{A?}$
        \item \color{lightgray} Bullet combinator $C_1 \bullet C_2$
    \end{itemize}
\end{frame}
\begin{frame}
    \frametitle{Example}
    Channel splitter:
    {\small $$*[C?c; [\overline{A?} \rightarrow A!c \, | \, \overline{B?} \rightarrow B?c]]$$}
\end{frame}
\subsection{CHP LTS}
\begin{frame}
    \frametitle{CHP LTS}
    \tiny
    $$
        % skip, assign, sequential and parallel composition rules
        \frac{\langle P, \sigma \rangle \xrightarrow{\delta} \langle P', \sigma' \rangle}{\langle P ; Q, \sigma \rangle \xrightarrow{\delta} \langle P' ; Q, \sigma' \rangle} \qquad
        \frac{\langle P, \sigma \rangle \xrightarrow{\delta} \langle P', \sigma' \rangle}{\langle P || Q, \sigma \rangle \xrightarrow{\delta} \langle P' || Q, \sigma' \rangle} \qquad
        \frac{\langle Q, \sigma \rangle \xrightarrow{\delta} \langle Q', \sigma' \rangle}{\langle P || Q, \sigma \rangle \xrightarrow{\delta} \langle P || Q', \sigma' \rangle} $$$$
        \frac{}{\langle \mathtt{skip}; P, \sigma \rangle \xrightarrow{\tau} \langle P, \sigma \rangle} \qquad
        \frac{}{\langle \mathtt{skip} || P, \sigma \rangle \xrightarrow{\tau} \langle P, \sigma \rangle} \qquad
        \frac{}{\langle P || \mathtt{skip}, \sigma \rangle \xrightarrow{\tau} \langle P, \sigma \rangle} $$$$
        \frac{\sigma \models \mathrm{b} = \ell}{\langle a := \mathrm{b}, \sigma \rangle \xrightarrow{\tau} \langle \mathtt{skip}, \sigma[a = \ell] \rangle} \qquad
        % selection rules
        \frac{\sigma \models \mathrm{b}_i = \mathtt{true}} {\langle [\mathrm{b}_1 \rightarrow P_1  | \; \ldots \; | \mathrm{b}_n \rightarrow P_n ] , \sigma \rangle \xrightarrow{\tau} \langle P_i , \sigma\rangle  } $$$$
        \frac{\sigma \models \mathrm{b}_1 = \mathtt{false} \wedge \; \ldots \; \wedge \; \mathrm{b}_{i-1} = \mathtt{false} \; \wedge \mathrm{b}_i = \mathtt{true} \wedge \mathrm{b}_{i+1} = \mathtt{false} \wedge \; \ldots \; \wedge \; \mathrm{b}_n = \mathtt{false}} {\langle [ \mathrm{b}_1 \rightarrow P_1  \talloblong \; \ldots \; \talloblong \mathrm{b}_n \rightarrow P_n ] , \sigma \rangle \xrightarrow{\tau} \langle P_i , \sigma\rangle  } $$$$
        \frac{\sigma \models b_1 = \mathtt{false} \wedge \ldots \wedge b_n = \mathtt{false}}{\langle *[ \mathrm{b}_1 \rightarrow P_1  \talloblong \; \ldots \; \talloblong \mathrm{b}_n \rightarrow P_n ] , \sigma \rangle \xrightarrow{\tau} \langle \mathtt{skip} , \sigma\rangle } \qquad
        \frac{\sigma \models b_1 = \mathtt{false} \wedge \ldots \wedge b_n = \mathtt{false}}{\langle *[ \mathrm{b}_1 \rightarrow P_1  | \; \ldots \; | \mathrm{b}_n \rightarrow P_n ] , \sigma \rangle \xrightarrow{\tau} \langle \mathtt{skip} , \sigma \rangle } $$$$
    %   \frac{\langle S, \sigma \rangle \xrightarrow{\tau} \langle P, \sigma \rangle}{\langle *S, \sigma \rangle \xrightarrow{\tau} \langle P;*S, \sigma \rangle} $$$$
        \frac{\sigma \models \mathrm{b}_1 = \mathtt{false} \wedge \; \ldots \; \wedge \; \mathrm{b}_{i-1} = \mathtt{false} \wedge \mathrm{b}_i = \mathtt{true} \wedge \mathrm{b}_{i+1} = \mathtt{false} \wedge \; \ldots \; \wedge \; \mathrm{b}_n = \mathtt{false}} {\langle *[ \mathrm{b}_1 \rightarrow P_1  \talloblong \; \ldots \; \talloblong \mathrm{b}_n \rightarrow P_n ] , \sigma \rangle \xrightarrow{\tau} \langle P_i; \; *[ \mathrm{b}_1 \rightarrow P_1  \talloblong \; \ldots \; \talloblong \mathrm{b}_n \rightarrow P_n ] , \sigma\rangle  } $$$$
        \frac{\sigma \models \mathrm{b}_i = \mathtt{true}} {\langle *[ \mathrm{b}_1 \rightarrow P_1  | \; \ldots \; | \mathrm{b}_n \rightarrow P_n ] , \sigma \rangle \xrightarrow{\tau} \langle P_i;\;*[ \mathrm{b}_1 \rightarrow P_1  | \; \ldots \; | \mathrm{b}_n \rightarrow P_n ] , \sigma\rangle  }
    $$
\end{frame}
\begin{frame}
    \small
    $$
        % communication rules
        \frac{\sigma \models \mathrm{b} = \mathtt{true}}{\langle \{A!\mathrm{b}\}, \sigma \rangle \xrightarrow{A!\mathtt{true}} \langle \mathtt{skip}, \sigma \rangle} $$$$
        \frac{\sigma \models \mathrm{b} = \mathtt{false}}{\langle \{A!\mathrm{b}\}, \sigma \rangle \xrightarrow{A!\mathtt{false}} \langle \mathtt{skip}, \sigma \rangle} $$$$
        \frac{}{\langle \{A?a\}, \sigma \rangle \xrightarrow{A?a : \mathtt{true}} \langle \mathtt{skip}, \sigma[a = \mathtt{true}] \rangle} $$$$
        \frac{}{\langle \{A?a\}, \sigma \rangle \xrightarrow{A?a : \mathtt{false}} \langle \mathtt{skip}, \sigma[a = \mathtt{false}] \rangle} $$$$
        \frac{\langle P, \sigma \rangle \xrightarrow{A!\ell} \langle P', \sigma \rangle \quad \langle Q, \sigma' \rangle \xrightarrow{A?a : \ell} \langle Q', \sigma'' \rangle}{\langle P || Q, \sigma \rangle \xrightarrow{\tau} \langle P' || Q', \sigma[a = \ell] \rangle} $$$$
        \frac{\langle P, \sigma \rangle \xrightarrow{A?a : \ell} \langle P', \sigma' \rangle \quad \langle Q, \sigma'' \rangle \xrightarrow{A!\ell} \langle Q', \sigma'' \rangle}{\langle P || Q, \sigma \rangle \xrightarrow{\tau} \langle P' || Q', \sigma[a = \ell] \rangle}
        %\frac{\langle P, \sigma \rangle \xrightarrow{\delta} \langle P', \sigma' \rangle \quad \langle Q, \sigma'' \rangle \xrightarrow{\delta'} \langle Q', \sigma''' \rangle \quad (\delta, \delta', \Sigma) \in \mathcal{M}}{\langle P || Q, \sigma \rangle \xrightarrow{\tau} \langle P' || Q', \sigma[\Sigma] \rangle}
    $$
\end{frame}
%\begin{frame}
%    where $\mathcal{M} \subseteq \mathbf{Labels} \times \mathbf{Labels} \times \mathbf{Assignments}$ is defined by:
%        $$\frac{A \in \mathbf{Channels}}{(\{A!\ell\}, \{A?a : \ell\}, \{[a = \ell]\}) \in \mathcal{M}}$$
%        $$\frac{A \in \mathbf{Channels}}{(\{A?a : \ell\}, \{A!\ell\}, \{[a = \ell]\}) \in \mathcal{M}}$$
%        $$\frac{A \in \mathbf{Channels} \quad (\delta, \delta', \Sigma) \in \mathcal{M}}{(\delta \cup \{A?a : \ell\}, \delta \cup \{A!\ell\}, \Sigma \cup \{[a = \ell]\}) \in \mathcal{M}}$$
%        $$\frac{A \in \mathbf{Channels} \quad (\delta, \delta', \Sigma) \in \mathcal{M}}{(\delta \cup \{A!\ell\}, \delta \cup \{A?a : \ell\}, \Sigma \cup \{[a = \ell]\}) \in \mathcal{M}}$$
%\end{frame}
\begin{frame}
    \frametitle{Example}
    Channel splitter:
    {\small $$*[C?c; [\overline{A?} \rightarrow A!c \, | \, \overline{B?} \rightarrow B?c]]$$}
\end{frame}
\begin{frame}
    \small
    \begin{align*}
        & C?c; [\overline{A?} \rightarrow A!c \, | \, \overline{B?} \rightarrow B?c] \\
        & \hspace{0.5cm} \xrightarrow{C?\mathtt{true}} [\overline{A?} \rightarrow A!c \, | \, \overline{B?} \rightarrow B?c] \\
        & \hspace{1.0cm} \xrightarrow{\tau} A!c \\
        & \hspace{1.5cm} \xrightarrow{A!\mathtt{true}} \mathtt{skip} \\
        & \hspace{1.0cm} \xrightarrow{\tau} B!c \\
        & \hspace{1.5cm} \xrightarrow{B!\mathtt{true}} \mathtt{skip} \\
        & \hspace{0.5cm} \xrightarrow{C?\mathtt{false}} [\overline{A?} \rightarrow A!c \, | \, \overline{B?} \rightarrow B?c] \\
        & \hspace{1.0cm} \xrightarrow{\tau} A!c \\
        & \hspace{1.5cm} \xrightarrow{A!\mathtt{false}} \mathtt{skip} \\
        & \hspace{1.0cm} \xrightarrow{\tau} B!c \\
        & \hspace{1.5cm} \xrightarrow{B!\mathtt{false}} \mathtt{skip}
    \end{align*}
\end{frame}
\subsection{CHP to HSE+ Translation}
\begin{frame}
    \frametitle{CHP to HSE+ Translation}
    \small
    \begin{align*}
        \overline{A!} & \Rightarrow (A_t \vee A_f) \wedge \neg A_a \\
        \overline{A?} & \Rightarrow A_a \wedge \neg (A_t \vee A_f) \\
        A!\mathrm{b} & \Rightarrow \left\{ \begin{matrix} [\mathrm{b} \rightarrow A_t := \mathtt{true} \talloblong \neg \mathrm{b} \rightarrow A_f := \mathtt{true}]; [A_a];\\ A_t := \mathtt{false}; A_f := \mathtt{false}; [\neg A_a] \\
                                                          [A_a]; [\mathrm{b} \rightarrow A_t := \mathtt{true} \talloblong \neg \mathrm{b} \rightarrow A_f := \mathtt{true}];\\ [\neg A_a]; A_t := \mathtt{false}; A_f := \mathtt{false} \end{matrix} \right. \\
        A?a & \Rightarrow \left\{ \begin{matrix} [A_t \vee A_f]; a := A_t; A_a := \mathtt{true};\\ [\neg A_t \wedge \neg A_f]; A_a := \mathtt{false} \\
                                                 A_a := \mathtt{true}; [A_t \vee A_f]; a := A_t;\\ A_a := \mathtt{false}; [\neg A_t \wedge \neg A_f] \end{matrix} \right. \\
    \end{align*}
\end{frame}
%\subsection{Four-Phase Handshake}
%\begin{frame}
%    \frametitle{Four-Phase Handshake}
%    For a dataless communication between two ends $A$ and $B$, the four-phase handshake looks like this:
%    \begin{align*}
%        \text{Active end: } & A \uparrow; [B]; A \downarrow; [\neg B] \\
%        \text{Passive end: } & [A]; B \uparrow; [\neg A]; B \downarrow
%    \end{align*}
%\end{frame}
%\begin{frame}
%    To extend the four-phase handshake to allow for transfer of data, we replace an acknowledgement variable $A$ or $B$ with a set of data transfer variables $A_i$ or $B_i$ (we need as many data transfer variables as we have possible values to send).
%    In the one-bit case, we replace either $A$ with $A_t$ and $A_f$, or $B$ with $B_t$ and $B_f$:
%    \begin{align*}
%        \text{Active end: } & (A_t \lor A_f) \uparrow; [B]; (A_t \lor A_f) \downarrow; [\neg B] \\
%        \text{Passive end: } & [A_t \lor A_f]; \text{(process data)}; B \uparrow; [\neg A_t \land \neg A_f]; B \downarrow \\ \\
%        \text{Active end: } & A \uparrow; [B_t \lor B_f]; \text{(process data)}; A \downarrow; [\neg B_t \land \neg B_f] \\
%        \text{Passive end: } & [A]; (B_t \lor B_f) \uparrow; [\neg A]; (B_t \lor B_f) \downarrow
%    \end{align*}
%\end{frame}
\section{Weak Bisimilarity}
\begin{frame}
\frametitle{Program Equivalence}
\begin{itemize}
\item Would like to cut up our the space into equivalent sets
\item Found that weak bisimilarity gave us enough flexibility while still being restrictive enough
\end{itemize}
\end{frame}
\subsection{Definition}
\begin{frame}
\frametitle{Weak Bisimulation}
\begin{itemize}
\item Intuitively: Does not care about actions on internal channels
\pause
\item Formally:
\begin{itemize}
\item A binary relation $\mathcal{S}$ over $\mathcal{P}$ is a weak simulation if, whenever $P \mathcal{S} Q$:
\begin{itemize}
\item if $P \overset{e}{\Rightarrow} P^\prime$ then there exists a $Q \in \mathcal{P}$ s.t. $Q  \overset{e}{\Rightarrow} Q^\prime$ and $P^\prime \mathcal{S} Q^\prime$
\item where $\Rightarrow$ is the reflexive transitive closure over $\rightarrow$, and $\overset{\alpha}{\Rightarrow} \overset{\triangle}{=} \Rightarrow \overset{\alpha}{\rightarrow} \Rightarrow$.
\end{itemize}
\item A binary relation $\mathcal{S}$ over $\mathcal{P}$ is a weak bisimulation if both $\mathcal{S}$ and its converse are weak simulations.
\item Weak bisimilarity, $\approx$ is the union of all weak bisimulations.
\end{itemize}
\end{itemize}
\end{frame}
\subsection{Applications}
\begin{frame}
\frametitle{Program Decomposition}
\begin{itemize}
\item If we have some CHP program of the form: $...;P;...$
\begin{itemize}
\item where $P$ is some program fragment:
\end{itemize}
\pause
\item The decomposition of that program into two separate programs is:
\begin{itemize}
\item $(...;C;...) || *[\overline{C} \rightarrow P; C] $ where C is some fresh channel.
\end{itemize}
\pause
\item If $P$ does not contain any uses of shared variables or channel actions:
\begin{itemize}
\item$(...;C;...) || *[\overline{C} \rightarrow C; P] $ where C is some fresh channel.
\end{itemize}
\end{itemize}
\end{frame}
\begin{frame}
\frametitle{Program Decomposition}
\begin{itemize}
\item We would like to show that $ ...;P;... \approx (...;C;...) || *[\overline{C} \rightarrow P; C]$
\pause
\item Equivalent to showing that $P \approx C || *[\overline{C} \rightarrow P; C]$
\pause
\item This will be the case, as the probe of C will not become true until the communication action begins, and then the only action possible are the ones that were possible in S.
\end{itemize}
\end{frame}
\begin{frame}
\frametitle{Process Decomposition}
\begin{itemize}
\item We would like to show that $ ...;P;... \approx (...;C;...) || *[\overline{C} \rightarrow C; P]$
\pause
\item This will only hold when the only actions in $P$ are $\tau$ actions
\pause
\item If that is the case, this will create one additional $\tau$ action, but do nothing else
\pause
\item Therefore, the two are weakly bisimilar
\end{itemize}
\end{frame}
\section{Conclusions}
\subsection{Summary}
\begin{frame}
\frametitle{Summary and Conclusions}
\begin{itemize}
\item Encoded a CSP analogue in a shared-variable language
\pause
\item Showed that we could guarantee implementability relations
\pause
\item Discussed the formal semantics of the languages
\pause
\item Showed some concurrency-increasing transformations that preserve weak bisimilarity
\end{itemize}
\end{frame}
\begin{frame}[plain]
\begin{center}
\Huge Questions?
\end{center}
\end{frame}
\end{document}
